% Options for packages loaded elsewhere
\PassOptionsToPackage{unicode}{hyperref}
\PassOptionsToPackage{hyphens}{url}
%
\documentclass[
]{article}
\usepackage{amsmath,amssymb}
\usepackage{iftex}
\ifPDFTeX
  \usepackage[T1]{fontenc}
  \usepackage[utf8]{inputenc}
  \usepackage{textcomp} % provide euro and other symbols
\else % if luatex or xetex
  \usepackage{unicode-math} % this also loads fontspec
  \defaultfontfeatures{Scale=MatchLowercase}
  \defaultfontfeatures[\rmfamily]{Ligatures=TeX,Scale=1}
\fi
\usepackage{lmodern}
\ifPDFTeX\else
  % xetex/luatex font selection
\fi
% Use upquote if available, for straight quotes in verbatim environments
\IfFileExists{upquote.sty}{\usepackage{upquote}}{}
\IfFileExists{microtype.sty}{% use microtype if available
  \usepackage[]{microtype}
  \UseMicrotypeSet[protrusion]{basicmath} % disable protrusion for tt fonts
}{}
\makeatletter
\@ifundefined{KOMAClassName}{% if non-KOMA class
  \IfFileExists{parskip.sty}{%
    \usepackage{parskip}
  }{% else
    \setlength{\parindent}{0pt}
    \setlength{\parskip}{6pt plus 2pt minus 1pt}}
}{% if KOMA class
  \KOMAoptions{parskip=half}}
\makeatother
\usepackage{xcolor}
\setlength{\emergencystretch}{3em} % prevent overfull lines
\providecommand{\tightlist}{%
  \setlength{\itemsep}{0pt}\setlength{\parskip}{0pt}}
\setcounter{secnumdepth}{-\maxdimen} % remove section numbering
\ifLuaTeX
  \usepackage{selnolig}  % disable illegal ligatures
\fi
\IfFileExists{bookmark.sty}{\usepackage{bookmark}}{\usepackage{hyperref}}
\IfFileExists{xurl.sty}{\usepackage{xurl}}{} % add URL line breaks if available
\urlstyle{same}
\hypersetup{
  hidelinks,
  pdfcreator={LaTeX via pandoc}}

\author{}
\date{}

\begin{document}

+++ title = ``Kings'' tags = {[}``philosophy''{]} date = ``2023-02-26''
author = ``Amar Shavir'' image = ``/kings.png'' +++

\hypertarget{what-is-the-role-of-a-king}{%
\section{What is the role of a king?}\label{what-is-the-role-of-a-king}}

The hierarchical position has evolved drastically throughout history and
across different cultures. Throughout history, the king was seen as a
divine or semi-divine figure maintaining ultimate authority over their
subjects. With the establishment of democracy in the fifth century
B.C.E. Athens the king transformed into a more mortal political leader
who governed with the support of a powerful noble class or with the
consent of a representative body. In many feudal societies, the king was
the supreme ruler who held all political power and was responsible for
administrating justice, maintaining order, and protecting the kingdom
from external threats. Because of the innate parasocial relationship
between kings and their subjects, or the more favorable term
``constituents,'' Kings were often considered divinely ordained or,
better yet, designated themselves to be of divine origin themselves.
This ``Godhood'' gave way to absolute power, as no mortal subject would
ever dare oppose a God. In more modern societies, the role of the king
has changed dramatically. In many countries, the monarchy is now a
ceremonial or symbolic institution with limited political power, and
elected representatives hold the real power. The symbolic nature of
royalty stands ever present through the United Kingdom, In other
countries however, the monarchy has been abolished altogether.

\hypertarget{what-are-the-idealistic-examples-of-kings-in-media}{%
\section{What are the idealistic examples of kings in
media?}\label{what-are-the-idealistic-examples-of-kings-in-media}}

The fine arts and media were birthed and enjoyed by the elites -- lords,
ladies, kings, queens -- of society. Subjects in the position of bards,
storytellers, and arts would make a living by portraying their king in
the best light to ``making a living'' or, more accurately, to continue
living. In any case, the progression of civilizations, stretching
outside the bounds of the obsolete feudal governance, idealizes kings as
being that should be noble and just. The relevant case of media
idealization would be the Great king Arthur (and the knights of the
round table). According to the legend, King Arthur Pendragon was a
legendary British king who ruled during the late 5th and early 6th
century. He was son of King Uther Pendragon and was raised by Sir Hector
as a ward of the court. As a young man, Arthur pulled the sword
Excalibur from a stone. A signification, he was the true king of
Britain, As King Arthur established a court of Camelot.

See \href{https://www.britannica.com/topic/King-Arthur}{King Arthur} For
more information about the legendary king.

\hypertarget{king-as-symbolism}{%
\section{King as symbolism}\label{king-as-symbolism}}

While the story of King Arthur is the most relevant when it comes to my
main point of this paper, outlining idealistic rulers and their
real-life counterparts. There is a noticeable inclination in modern
civilizations to worship those who hold the position of a celebrity.

Musicians, Actors, and now the recent more recent development of online
influencers carry with them the same parasocial relationship as the
kings and queens of feudal times. This relationship between celebrities
and fans is, in part, in some way exploitive, whether that be for more
money, ever an increasing political agenda, or some other social metric;
this ``exploitation'' does happen to drastically vary from celebrity to
celebrity as the level of opportunistic capacity varies from person to
person, but I digress.

In 2018, the ``Black Panther'' film was released, based on the character
and directed by Ryan Coogler. The film was set in the Marvel Cinematic
Universe. It starred Chadwick Boseman as T'Challa/Black Panther, Michael
B. Jordan as Erik Killmonger, Lupita Nyong'o as Nakia, and Danai Gurira
as Okoye. The film was a critical and commercial success and received
widespread praise for its groundbreaking representation of black
characters and themes.

It was also the first superhero film nominated for the Academy Award for
Best Picture. The film inspired thousands to millions of young people of
color that they could be their king or queen. As anecdotal evidence, I
was one of them. When I first left the movie theater at 14, I was
utterly filled with hope and some pride because the people in the
production ``looked just like me''. In this instance, both the actor and
character take the role of pseudo-kings; for the minds of youth, having
a robust role model is a crucial part of developing into an adult.\\
However, they are nothing more than public figures. These pseudo-kings
hold power in mind and not by will, thus making them symbolic of
whatever cause they choose part take in. And depending on the cause, it
can also be their downfall as a symbol.

Looking back, it is strange that I had such a strong reaction to the
movie when the character itself was not real, and none of the events
happened. There are far better role models than a King living in an
isolated society fighting crime in a catsuit sponsored by a magical
space rock (Yes, Vibranium is, in fact, magic, it does way too much
crazy nonsense not to be, I do not care that they call it a metal).
Better role models for young black Americans would be Malcolm X, MLK,
Muhammad Ali, or more contemporary leaders such as Magatte wade, David
Goggins, or Idris Elba (Yes, Idris Elba is an actor, but he is also a
film developer in the west African nation of Ghana, so I am making an
exception).

I believe it is essential to clarify that these sorts of pseudo-kings
and role models should be recognized as separate entities. Role models
are essential in inspiring others in the face of hardship; false kings
do nothing more than leach off the people, positing them at their status
level with no real power over the public besides drawing their
attention. While I provided examples of famous people as Role models,
those two traits do not have to coexist simultaneously. A role model can
be anyone, but as an idealistic model for behavior, you should avoid
meeting them or letting their actions affect you directly. If the
actions of a public figure do directly influence you, their role
transforms into that of a pseudo-king.

\hypertarget{vlad-dracula-tepes}{%
\section{Vlad Dracula Tepes}\label{vlad-dracula-tepes}}

The noble king is often found in media but rarely in history, as a dark
mirror to the idealism of king Arthur is the realism of Vlad II. Vlad
II, also known as Vlad the Impaler, was a medieval ruler of Wallachia, a
region in present-day Romania, who lived from 1431 to 1476. Vlad is
known for his cruelty and his use of impalement as a method of
execution, which earned him the nickname ``Vlad the Impaler''.

Vlad's father was a member of the Order of the Dragon, a chivalric order
founded to protect Christianity in Europe. After his father's death,
Vlad became the ruler of Wallachia, a position he held three times
between 1448 and 1476. During his reign, Vlad was known for his brutal
tactics against his enemies, including but not limited to\ldots{} ``\,''
impaling them on stakes and leaving their bodies to rot in public view
as warning to others. Boiling, burning, and blinding. ``\,''\\
Some saw Vlad as a hero for his efforts to resist the Ottoman Empire,
which was seeking to expand into Eastern Europe. He is mainly known for
his campaign against the Ottoman Sultan Mehmed II, during which he
impaled tens of thousands of Ottoman soldiers and prisoners of war. Vlad
was eventually captured by his enemies and imprisoned, and he was killed
in battle in 1476 while trying to regain his throne. Today, he is
remembered as a controversial figure in Romanian history, and his legacy
continues to be debated and interpreted.

Vlad Tepe's story is the public perception that ``Dracula'' has. Vlad
III has been mention as a part of the violent tyrants of all time for
his ruthlessness and cruelty. A large part of Vlad's perception has to
do a cycle of propaganda.

Vlad maintained a deep-seated hatred of the Valachian nobles who
murdered his father and brother. The moment Vlad regained power, he
quickly disposed of the traitorous noblemen. Naturally, the nobles
retaliated by spreading false rumors and plotting to depose him. With
the aid of a fake letter, they managed to convince the king of Hungary,
Matthias I, that Vlad desired to ally himself with the Sultan. As a
result, he imprisoned the Wallachian voivode in Visegrád. He was killed
fighting the nobles after he regained the throne. ``During Vlad's
imprisonment, news of his false betrayal spread through Europe, fueled
further by his reputation. He became the subject of several paintings
where he is portrayed as either
\href{https://www.thecollector.com/pontius-pilate-the-man-who-sentenced-jesus-christ-to-death/}{Pontius
Pilate} or as Aegeas, the Roman proconsul of Patras, present at the
crucifixion of Saint Andrew. Also, he is the subject of a poem written
by Michael Beheims, entitled \emph{Story of a Despot Called Dracula,
Voivode of Wallachia}.'' It should be clear by now that regardless of
how cruel he may or may not have been, that was the reputation Vlad III
garnished. Like I had previously mentioned, the Vlad II was crucial in
resisting The Ottoman Empire march on to West Europe and, because of
this, is responsible for protecting Christianity as we know it now. The
same Christianity that would go on to label him as Tepes as Dracula.

\hypertarget{a-hero-is-a-monster-for-the-week}{%
\section{A hero is a monster for the
week}\label{a-hero-is-a-monster-for-the-week}}

King Arthur, I would argue, did nothing more than pickup a sword and
defend his people from the monstrous invaders. In the same way, Vlad III
picked up ten foot wooden stakes and did the same thing. There are two
main differences from what Arthur and Dracula story. Difference A is
that the monsters that Arthur killed were just that \emph{monsters},
that don't exist in reality and are only a figment of our shared
imagination.

To add depth to the character of Dracula, the vast majority of the
populace claim that the primary inspiration for the vampire king was the
real-life Vladimir Dracul Tepes. This basis, however, has no real merit
for a couple of reasons. In Bram Stoker's tale, Dracula follows none of
the actions or even acts similarly to Vlad Tepes, other than being
``bloodthirsty'' but to claim that Stoker's only reference to a Tepes
was taking an abstract adjective and making it hyper-literal.

\hypertarget{the-dracula-tepes-in-mediaanime}{%
\section{The Dracula Tepes in
media/anime}\label{the-dracula-tepes-in-mediaanime}}

While Dracula is frequently represented in the film as a direct
representation of Bram Stoker's vampiric tyrant, there are a few
extraordinary retellings. My favorite is not a ``Dracula,'' but one that
fits the ``tyrannical blood lusted king archetype''.

King Bradley is a pivotal character in the Fullmetal Alchemist and
Fullmetal Alchemist Brotherhood anime series. As the leader of the
country of Amestris and holder of the title ``Führer,'' Bradley is
portrayed as a charismatic, powerful, and ruthless leader who enforces
the government's laws with an iron fist. However, as the story
progresses, his true identity as a homunculus is revealed, adding depth
and complexity to his character and his motivations. One of the most
significant aspects of King Bradley's character is his role as a leader.
Throughout the series, he is shown as a charismatic and effective leader
who commands respect and loyalty from his subordinates. He is also
skilled in the art of combat, possessing incredible speed, agility, and
strength. As a result, he is often called upon to carry out dangerous
missions and defend his country from external threats. However, as the
story progresses, it becomes clear that Bradley's leadership is not as
benevolent as it initially appears. He is revealed to be a homunculus
created through alchemy with no soul. This revelation adds complexity to
his character, as it raises questions about his true loyalties and
motivations. It is eventually revealed that Bradley's ultimate goal is
to carry out the plans of his creator, Father, to create a massive
transmutation circle that will grant him god-like power.

One of King Bradley's most notorious actions is his role in the Ishvalan
War, a conflict between the state of Amestris and the Ishvalan people.
Bradley was directly involved in the war as a commanding officer, and
his actions during this time were brutal and merciless. He led the
military's efforts to eliminate the Ishvalans, using alchemical weapons
and ordering his soldiers to kill indiscriminately. As a result, many
Ishvalans were massacred, and the survivors were forced into ghettos.
Bradley's cruelty extended beyond the Ishvalan War, as he was also
responsible for numerous assassinations and acts of terror against his
own people. He ordered the murder of anyone who threatened his power or
the stability of the state, including members of his own cabinet. He was
also directly involved in the creation of the Homunculi, artificial
beings created through alchemy, and used them as tools to carry out his
schemes and eliminate his enemies. Despite his ruthless nature and
questionable motivations, Bradley is shown to have a human side. He is
portrayed as having a deep affection for his wife, whom he refers to as
``Mrs.~Bradley,'' and his son, Selim. This aspect of his character
creates a conflict between his duties as a leader and his personal
relationships, as he is forced to choose between his duty to his country
and his duty to his family.Overall, King Bradley's reign was marked by
violence, oppression, and terror, and he was willing to commit any
atrocity to maintain his power and control. His actions serve as a
cautionary tale about the dangers of unchecked power and the importance
of accountability and justice in leadership.

\hypertarget{the-context-of-kings}{%
\section{The context of Kings}\label{the-context-of-kings}}

This section is in no way a defense of the actions of people with the
role of the position of king. Unlike most, royalty is granted a level of
self-autonomy that most of humanity did not have. With that said, life
before the 20th century was unapologetically terrible for everyone. Even
with significantly lower economic power, the quality of life that the
``average'' westerner experience is significantly higher than the kings
in history.

History has been a mixed bag for different individuals and groups, as it
has been marked by both progress and setbacks, triumphs and tragedies,
and achievements and atrocities. While some people have enjoyed great
wealth, power, and privilege, others have experienced poverty,
oppression, and discrimination.

Some of the factors that have contributed to the difficulties faced by
people throughout history include:

\begin{enumerate}
\def\labelenumi{\arabic{enumi}.}
\item
  Power imbalances: Throughout history, power has often been
  concentrated in the hands of a small ruling elite, leading to
  widespread inequality and exploitation of the masses.
\item
  Conflict and war: Conflicts and wars have ravaged societies throughout
  history, leading to loss of life, displacement, and destruction of
  property.
\item
  Disease and famine: Epidemics and famines have plagued societies
  throughout history, causing widespread suffering and death.
\item
  Natural disasters: Natural disasters such as earthquakes, floods, and
  hurricanes have caused widespread destruction and loss of life
  throughout history.
\item
  Prejudice and discrimination: Throughout history, individuals and
  groups have faced discrimination and prejudice based on factors such
  as race, ethnicity, religion, gender, and sexuality.
\end{enumerate}

While there have been many challenges throughout history, it's also
worth noting that there have been many positive developments and
achievements as well, such as advances in science and technology,
improvements in healthcare and education, and the establishment of
democratic institutions and civil rights.

Because of the institution of succession, most kings were born into
their power. A result of hyperactive nepotism is a lot of unqualified
leaders. And even those who would be considered to be qualified still
often find themselves in the pitfalls of history. King Charles I was the
monarch of England, Scotland, and Ireland from 1625 until his execution
in 1649. He was born on November 19, 1600, at Dunfermline Palace in
Scotland, the second son of James VI of Scotland and I of England.
Charles was raised in the Anglican faith and received a royal education.
Charles became king in 1625 after his father's death, and he quickly
became embroiled in conflict with Parliament over issues such as
taxation, foreign policy, and religious freedom. In 1642, tensions
between Charles and Parliament erupted into the English Civil War, with
Charles leading the Royalist forces against the Parliamentarian forces
led by Oliver Cromwell. The war was long and brutal, with both sides
committing atrocities and inflicting significant damage on the country.
In 1646, Charles surrendered to the Scottish army, but the Scottish
turned him over to Parliament in exchange for payment of their expenses.
In 1648, Charles tried to regain his throne by force, but the
Parliamentarians defeated and captured him. In January 1649, he was put
on trial for high treason, found guilty, and sentenced to death. On
January 30, 1649, Charles was beheaded at the Banqueting House in
Whitehall, London. Charles' execution was a watershed moment in English
history, leading to the establishment of the Commonwealth of England,
which lasted until the Restoration of the monarchy in 1660. Charles was
later canonized by the Church of England as a martyr and is commemorated
in the Church's calendar on January 30.

The Royal Uk writes this about Charles\ldots{}

``\,``\,'' Charles was reserved (he had a residual stammer),
self-righteous and had a high concept of royal authority, believing in
the divine right of kings. He was a good linguist and a sensitive man of
refined tastes. He spent a lot on the arts, inviting the artists Van
Dyck and Rubens to work in England, and buying a great collection of
paintings by Raphael and Titian (this collection was later dispersed
under Cromwell). Charles I also instituted the post of Master of the
King's Music, involving supervision of the King's large band of
musicians; the post survives today. ``\,``\,''

(It turns out Grammarly is no a fan of British English)

Charles, the first, is a character I do not believe to be an outlier in
the history of kings. In fact, I believe His rise and fall are far more
representative of the history of kings than figures like King Cyrus or
Alexander. That is to say, being placed on a high pedestal in society
creates a level of exposure for one individual that a person is not
naturally equipped to handle. Charles, the first, only cared about the
purpose and his own interest, and because of the level of power that a
king has, he could do what he wanted without any resistence. He failed
to calculate the effects of his actions or how the public would view
him, and due to that failure lost his life

\hypertarget{be-wary-of-those-who-aspire-to-be-king.}{%
\section{Be wary of those who aspire to be
``king.''}\label{be-wary-of-those-who-aspire-to-be-king.}}

a cautionary note about those who seek positions of power and authority.
While not all individuals who aspire to leadership roles are inherently
bad or corrupt, history has shown that many leaders have abused their
power and committed atrocities in the pursuit of their own interests.
Leaders who seek power for its own sake may be more interested in
maintaining their own position and authority than in serving the
interests of their constituents. They are willing to compromise their
values, to lie, to cheat, or commit acts of violence to protect their
position or to further their own agenda. Therefore, it is important to
be vigilant and discerning when evaluating those who seek leadership
roles. It is important to consider their values, track record, and
intentions, as well as their ability to work collaboratively and listen
to diverse perspectives. Ultimately, leaders should be held accountable
for their actions and should be motivated by a genuine desire to serve
their community and to make positive changes. By being cautious and
vigilant about those who seek power, we can help to ensure that our
leaders act in the best interests of their constituents and society as a
whole. Make no mistake my western readers, we have kings in our world
yet, and the most powerful ones are the ones that you do not know exist.

\end{document}
